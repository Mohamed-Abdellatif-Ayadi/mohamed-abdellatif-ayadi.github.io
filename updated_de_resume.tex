\documentclass[10pt, letterpaper]{article}

% Packages:
\usepackage[
    ignoreheadfoot, % set margins without considering header and footer
    top=2 cm, % seperation between body and page edge from the top
    bottom=2 cm, % seperation between body and page edge from the bottom
    left=2 cm, % seperation between body and page edge from the left
    right=2 cm, % seperation between body and page edge from the right
    footskip=1.0 cm, % seperation between body and footer
    % showframe % for debugging 
]{geometry} % for adjusting page geometry
\usepackage{titlesec} % for customizing section titles
\usepackage{graphicx} 
\usepackage{tabularx} % for making tables with fixed width columns
\usepackage{array} % tabularx requires this
\usepackage[dvipsnames]{xcolor} % for coloring text
\usepackage{fontawesome5}
\usepackage{xcolor}
\definecolor{portfolioColor}{RGB}{0, 102, 204} % nice strong blue

\definecolor{primaryColor}{RGB}{0, 0, 0} % define primary color
\usepackage{enumitem} % for customizing lists
\usepackage{fontawesome5} % for using icons
\usepackage{amsmath} % for math
\usepackage[
    pdftitle={Mohamed Abdellatif Ayadi's CV},
    pdfauthor={Mohamed Abdellatif Ayadi},
    pdfcreator={LaTeX with RenderCV},
    colorlinks=true,
    urlcolor=primaryColor
]{hyperref} % for links, metadata and bookmarks
\usepackage[pscoord]{eso-pic} % for floating text on the page
\usepackage{calc} % for calculating lengths
\usepackage{bookmark} % for bookmarks
\usepackage{lastpage} % for getting the total number of pages
\usepackage{changepage} % for one column entries (adjustwidth environment)
\usepackage{paracol} % for two and three column entries
\usepackage{ifthen} % for conditional statements
\usepackage{needspace} % for avoiding page brake right after the section title
\usepackage{iftex} % check if engine is pdflatex, xetex or luatex
\usepackage{parskip} % For spacing between paragraphs
\usepackage[margin=1cm]{geometry}

\usepackage{tikz} % For circular clipping

% Ensure that generate pdf is machine readable/ATS parsable:
\ifPDFTeX
    \input{glyphtounicode}
    \pdfgentounicode=1
    \usepackage[T1]{fontenc}
    \usepackage[utf8]{inputenc}
    \usepackage{lmodern}
\fi

\usepackage{charter}

% Some settings:
\raggedright
\AtBeginEnvironment{adjustwidth}{\partopsep0pt} % remove space before adjustwidth environment
\pagestyle{empty} % no header or footer
\setcounter{secnumdepth}{0} % no section numbering
\setlength{\parindent}{0pt} % no indentation
\setlength{\topskip}{0pt} % no top skip
\setlength{\columnsep}{0.15cm} % set column seperation
\pagenumbering{gobble} % no page numbering

\titleformat{\section}{\needspace{4\baselineskip}\bfseries\large}{}{0pt}{}[\vspace{1pt}\titlerule]

\titlespacing{\section}{
    % left space:
    -1pt
}{
    % top space:
    0.3 cm
}{
    % bottom space:
    0.2 cm
} % section title spacing

\renewcommand\labelitemi{$\vcenter{\hbox{\small$\bullet$}}$} % custom bullet points
\newenvironment{highlights}{
    \begin{itemize}[
        topsep=0.10 cm,
        parsep=0.10 cm,
        partopsep=0pt,
        itemsep=0pt,
        leftmargin=0 cm + 10pt
    ]
}{
    \end{itemize}
} % new environment for highlights

\newenvironment{highlightsforbulletentries}{
    \begin{itemize}[
        topsep=0.10 cm,
        parsep=0.10 cm,
        partopsep=0pt,
        itemsep=0pt,
        leftmargin=10pt
    ]
}{
    \end{itemize}
} % new environment for highlights for bullet entries

\newenvironment{onecolentry}{
    \begin{adjustwidth}{
        0 cm + 0.00001 cm
    }{
        0 cm + 0.00001 cm
    }
}{
    \end{adjustwidth}
} % new environment for one column entries

\newenvironment{twocolentry}[2][]{
    \onecolentry
    \def\secondColumn{#2}
    \setcolumnwidth{\fill, 4.5 cm}
    \begin{paracol}{2}
}{
    \switchcolumn \raggedleft \secondColumn
    \end{paracol}
    \endonecolentry
} % new environment for two column entries

\newenvironment{threecolentry}[3][]{
    \onecolentry
    \def\thirdColumn{#3}
    \setcolumnwidth{, \fill, 4,5 cm}
    \begin{paracol}{3}
    {\raggedright #2} \switchcolumn
}{
    \switchcolumn \raggedleft \thirdColumn
    \end{paracol}
    \endonecolentry
} % new environment for three column entries

\newenvironment{header}{
    \setlength{\topsep}{0pt}\par\kern\topsep\centering\linespread{1.5}
}{
    \par\kern\topsep
} % new environment for the header

\newcommand{\placelastupdatedtext}{% \placetextbox{<horizontal pos>}{<vertical pos>}{<stuff>}
  \AddToShipoutPictureFG*{% Add <stuff> to current page foreground
    \put(
        \LenToUnit{\paperwidth-2 cm-0 cm+0.05cm},
        \LenToUnit{\paperheight-1.0 cm}
    ){\vtop{{\null}\makebox[0pt][c]{
        \small\color{gray}\textit{Zuletzt aktualisiert im September 2024}\hspace{\widthof{Zuletzt aktualisiert im September 2024}}
    }}}%
  }%
}%

% save the original href command in a new command:
\let\hrefWithoutArrow\href

\begin{document}
    \newcommand{\AND}{\unskip
        \cleaders\copy\ANDbox\hskip\wd\ANDbox
        \ignorespaces
    }
    \newsavebox\ANDbox
    \sbox\ANDbox{$|$}

    \begin{header}
   \begin{center}
    % Circular photo using tikz
    \begin{tikzpicture}
        \clip (1.4,2.8) circle(2.4cm); % Circle with a radius of 2.5cm
        \node at (1,1) {\includegraphics[width=7.5cm]{image-cv.jpg}}; % Adjust the width
    \end{tikzpicture}

    \vspace{0.5cm} % Add spacing between the photo and text
    {\fontsize{20pt}{24pt}\selectfont \textbf{MOHAMED ABDELLATIF AYADI}}
\end{center}
        
\vspace{5 pt}

    \normalsize
    \textit{Student der Informatik (B.Sc.) an der Technischen Universität Dortmund}

        \vspace{5 pt}

      \faEnvelope\ \href{mailto:mohamed.ayadi.data@gmail.com}{\textcolor{primaryColor}{mohamed.ayadi.data@gmail.com}} \quad
\faPhone\ \href{tel:+49-152-5230-1739}{\textcolor{primaryColor}{0152 5230 1739}} \quad
\faGithub\ \href{https://github.com/Mayedi007}{\textcolor{primaryColor}{github.com/Mayedi007}} \quad
\faGlobe\ \href{https://personal-portfolio-mohamedayadidat.replit.app}{\textcolor{portfolioColor}{Personal Portfolio}} \quad

\faLinkedin\ \href{https://linkedin.com/in/mohamed-abdellatif-ayadi}{\textcolor{primaryColor}{linkedin.com/in/mohamed-abdellatif-ayadi}}

    \end{header}

    \vspace{1cm}

\section*{Über mich}

Ich bin Mohamed Abdellatif Ayadi, Informatikstudent an der Technischen Universität Dortmund mit praktischer Erfahrung als Werkstudent im Vertrieb sowie in der Softwareentwicklung.

Vertrieb mit Leidenschaft, Studium mit Ehrgeiz, sozialer Einsatz mit Herz – dieses Motto begleitet mich sowohl akademisch als auch beruflich.

Mit großer Begeisterung für Softwareentwicklung, Künstliche Intelligenz und technologische Innovationen strebe ich aktuell eine Werkstudentenstelle an, um meine Kenntnisse praxisnah zu vertiefen und mich gezielt in den Bereichen Entwicklung oder IT-Consulting weiterzuentwickeln.

\section{Akademische Ausbildung}

\begin{twocolentry}{
      voraussichtlich 2027
}
    \textbf{Technische Universität Dortmund}, B.Sc. Informatik

    \vspace{0.10 cm}
   
        
       \end{twocolentry}
             
             
\vspace{0.2 cm}
\begin{twocolentry}{
    September 2022
}
    \textbf{Goethe-Institut Düsseldorf} \\
   Besuch eines intensiven Deutsch-Sprachkurses mit Abschluss auf C1-Niveau 
\end{twocolentry}
\vspace{0.2cm}

\begin{twocolentry}{
    Juli 2021, Sfax, Tunesien}

    \textbf{Pioneer High School of Sfax (Lycée Pilote de Sfax)}\\

   
     
             Abitur im Fach Mathematik 


\end{twocolentry}

    
    \section{Berufserfahrung}
        
        \begin{twocolentry}{
            April 2024 – Heute 
        }
            \textbf{Werkstudent}, Iperceramica Deutschland GmbH, \end{twocolentry}

        \vspace{0.10 cm}
        \begin{onecolentry}
            \begin{highlights}
                \item Nutzung von SAP und SAP S/4HANA zur Optimierung von Beständen, Aufträgen, Lieferprozessen und der Bearbeitung von Reklamationen.

               \item Vertrieb: Aktive Kundengewinnung im B2B- und B2C-Bereich durch gezielte Akquise und Aufbau langfristiger Kundenbeziehungen. Verkauf und Beratung von hochwertigen Fliesen, Parkett, Sanitär und Badezimmermöbeln.

                \item Pflege von Kundendaten und Partnerbeziehungen in CRM und PRM-Systemen, zur Verbesserung der Kommunikation und Zusammenarbeit.
            \end{highlights}
        \end{onecolentry}

        \vspace{1 cm}
 
        \begin{twocolentry}{
            Oktober 2023 – April 2024
        }
            \textbf{Studentische Hilfskraft}, Technische Universität Dortmund, \end{twocolentry}

       \vspace{0.10 cm}
\begin{onecolentry}
    \begin{highlights}
        \item Übernahme der Position eines studentischen Tutors für den Kurs "Datenstrukturen, Algorithmen und Programmierung 1" als Minijob.
        \item Organisation und Durchführung von Tutorien für Erstsemester-Studierende mit Fokus auf objektorientierte Programmierung in Java.
        \item Aufgaben umfassten praktische Programmierübungen, Vertiefung und Erweiterung der Vorlesungsinhalte, Hausaufgabenbetreuung sowie gezielte Vorbereitung auf Prüfungen und Strategieplanung.
    \end{highlights}
\end{onecolentry}
    
\section{Projekte}

\begin{itemize}[leftmargin=*]
\item \textbf{Entwicklung eines KI-gestützten Chatbots}:  
Integration der OpenAI API in mein persönliches Portfolio (\href{https://personal-portfolio-mohamedayadidat.replit.app/chat}{Chatbot ansehen}), um ein interaktives Nutzererlebnis zu schaffen und praktische Fähigkeiten in API-Integration und Webentwicklung unter Beweis zu stellen.

\item \textbf{Flash Sale Plattform}:  
Entwicklung einer skalierbaren Backend-Plattform zur Verwaltung von Flash-Sales-Ereignissen mit hoher Benutzerlast. Fokus auf Datenmodellierung, Transaktionssicherheit und Performance-Optimierung (\href{https://github.com/Mayedi007/flash-sale-platform}{GitHub-Projekt ansehen}).

\item \textbf{Reddit Data Streaming Pipeline}:  
Implementierung einer Echtzeit-Datenpipeline zur Erfassung und Analyse von Reddit-Datenströmen. Einsatz moderner Data Engineering-Technologien (\href{https://github.com/Mayedi007/reddit-data-streaming-pipeline}{GitHub-Projekt ansehen}).

\item \textbf{Teilnahme am Kaggle-Wettbewerb "Corporación Favorita Grocery Sales Forecasting"}:  
Entwicklung eines Frameworks für On-the-Fly Data Augmentation zur Verbesserung der Prognosegenauigkeit bei saisonalen Schwankungen und unvollständigen Daten.

\item \textbf{Entwicklung eines UML-Visualisierungs-Plugins}:  
Erstellung eines Eclipse-Plugins zur Analyse und Visualisierung der Architektur eines Flugmanagementsystems. Fokus auf die Identifikation kritischer Modulabhängigkeiten und die Verbesserung der Systemwartbarkeit.
\end{itemize}

\vspace{0.2 cm}

\section{Fähigkeiten}

\textbf{Generative KI und Künstliche Intelligenz:}  
Entwicklung von KI-Anwendungen mit OpenAI-API, Chatbot-Entwicklung, Data Augmentation, Deep Learning, Forecasting, Maschinelles Lernen.

\textbf{Softwareentwicklung und Backend-Technologien:}  
Objektorientierte Programmierung (OOP), Java, Python, C-Programmierung, Spring Boot, Docker, API-Entwicklung, Datenbankdesign (MySQl).

\textbf{Datenengineering und Datenanalyse:}  
Echtzeit-Datenpipelines, Data Warehousing, Zeitreihenanalyse (Time Series Forecasting), ETL-Prozesse, Grundkenntnisse in Big Data.

\textbf{Werkzeuge und Plattformen:}  
GitLab, Eclipse, Microsoft Visual Studio, Linux-Systemadministration.

\textbf{Vertrieb und Business-Kompetenzen:}  
SAP, SAP S/4HANA, CRM-Systeme, Kundenakquise B2B/B2C, Kommunikation und Beziehungsmanagement.

\section{Sprachkenntnisse}

\begin{onecolentry}
    \textbf{Deutsch:} Fließend bis verhandlungssicher \\
    \textbf{Englisch:} Fließend bis verhandlungssicher \\
    \textbf{Französisch:} Muttersprache \\
    \textbf{Arabisch:} Muttersprache \\
    \textbf{Italienisch:} Basiskenntnisse
\end{onecolentry}

\section{Interessen}

Begeisterter Leser von Fachbüchern über Finanzen, Technologie und Innovationen.  
Regelmäßiger Besuch von Messen und Veranstaltungen, um technologische Neuheiten zu entdecken und mein Wissen über aktuelle Entwicklungen zu erweitern.  
Erfahrungen und Eindrücke aus diesen Besuchen teile ich regelmäßig in Beiträgen auf LinkedIn.  
Sportlich aktiv, insbesondere durch regelmäßiges Padelspielen.
    
\end{document}